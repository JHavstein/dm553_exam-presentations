\section{Finite automata and regular languages}

\subsection*{Outline}

\begin{itemize}
    \item Definition af finite automata (DFA/NFA)
    \item Bevis-skitse over ækvivalens af DFAs og NFAs
    \item Definition af regulære sprog og regular expressions
    \item Eksempel på en regular expression med en ækvivalent DFA
    \item Bevis for Pumping Lemmaet for regulære sprog
    \item Eksempel på anvendelse af pumping lemmaet 
\end{itemize}

\subsection*{Finite automata (NFA/DFA)}

\begin{itemize}
    \item Abstrakt model af en computer med ekstremt begrænset hukommelse
    \item Deterministisk finite automaton (DFA): 5-tuple $(Q,\Sigma,\delta,q_0,F)$, hvor
    \begin{itemize}
        \item $Q$: states 
        \item $\Sigma$: alfabet 
        \item $\delta$: transitionsfunktion, $\delta: \Sigma \rightarrow Q$
        \item $q_0 \in Q$: start-state 
        \item $F \subseteq Q$: accept-state 
    \end{itemize}
    \item Non-deterministisk finite automata (NFA): Somme som DFA pånær at $\delta: Q \times \Sigma_{\varepsilon} \rightarrow \mathcal{P}(Q)$ (d.v.s. når NFAen er i state $q \in Q$ og læser symbol $s \in \Sigma_{\varepsilon}$ kan NFAen lave en transition til multiple states).
\end{itemize}
\textbf{Thm.} NFAer og DFAer genkender den samme klasse af sprog. \\

\textbf{Bevis-skitse.} Enhver DFA er en NFA fordi $\forall q \in Q: q \subseteq \mathcal{P}(Q)$ (d.v.s. $\delta$ for DFAer er også gyldig for en NFA). \\

Enhver NFA har en ækvivalent DFA. Givet en NFA $N=(Q,\Sigma,\delta,q_0,F)$ kan vi konstruere en DFA $D=(Q', \Sigma, \delta',q_0',F')$, hvor $Q'=\mathcal{P}(Q)$, $q_0'=\{q_0\}$, $F'=\{R \in Q' | \; R \text{ indeholder en accept-state } N\}$. Da $Q'=\mathcal{P}(Q)$ kan vi nemt gøre $\delta'$ ækvivalent med $\delta$ ved for en given state og symbol $(q,r) \in Q \times \Sigma_{\varepsilon}$ at holde styr på, hvor $\delta$ kan lave transitioner til. Dette vil være $Q'' \subseteq \mathcal{P}(Q)$ så vi designer $\delta'$ således at denne reflekterer dette. \\

Når der er $\varepsilon$-transitioner i $N$ er der lidt flere trin forbundet med at konstruere $q_0'$, $\delta'$, og $F'$ men ækvivalensen holder stadig.
\begin{flushright}
    \qed 
\end{flushright}

\subsection*{Regular sprog og regular expressions}

\begin{itemize}
    \item Lad $L$ være et sprog. Så gælder: 
    \begin{itemize}
    	\item $L$ er regulært $\Leftrightarrow$ $L$ genkendes af en DFA.
    	\item $L$ er regulært $\Leftrightarrow$ $\exists$ en regular expression, der beskriver $L$
    \end{itemize}
    
    \item $R$ er en regular expression hvis $R$ er:
    \begin{itemize}
        \item $a$, for et $a \in \Sigma$,
        \item $\emptyset$ (det tomme sprog),
        \item $\varepsilon$ (sproget indeholdende den tomme streng), 
        \item $(R_1 \cup R_2)$ ($R_1,R_2$: regular expressions),
        \item $(R_1 \circ R_2)$ ($R_1,R_2$: regular expressions), eller 
        \item $(R_1^*)$ ($R_1$: regular expression)
    \end{itemize}
\end{itemize}

\textbf{Eksempel.} Lad $R=((ba \cup ab)^*b)$. $R$ er en regular expression og derfor er $((ba \cup ab)^*b)$ et regulært sprog. Den følgende DFA $D=(\{q_1,q_2,q_3,q_4\},\{a,b\},\delta,q_1,\{q_4\})$ har $L(D)=R$:
\begin{center}
    \begin{tikzpicture}[shorten >=1pt,node distance=2cm,on grid,auto] 
    \node[state,initial] (1) {$q_1$};
    \node[state] (2) [right=of 1] {$q_2$};
    \node[state] (3) [above=of 1] {$q_3$};
    \node[state,accepting] (4) [below=of 1] {$q_4$};
    \path[->]
        (1) edge [bend left=15] node {$b$} (2)
            edge [bend left=15] node {$a$} (3)
            edge node {$b$} (4)
        (2) edge [bend left=15] node {$a$} (1)
        (3) edge [bend left=15] node {$b$} (1);
    \end{tikzpicture}
\end{center}

\subsection*{Pumping lemma for regulære sprog}

Pumping lemmaet kan benyttes til at bevise at et sprog \textit{ikke} er regulært. \\

\textbf{Thm.} Hvis $A$ er et regulært sprog så $\exists p$ (pumpe-længde) s.t. hvis $s \in A$ og $|s|\ge p$ så kan $s$ deles op i tre dele $s=xyz$ således at $s$ tilfredsstiller:
\begin{enumerate}
    \item $\forall i \ge 0: xy^i z \in A$
    \item $|y|>0$
    \item $|xy| \le p$
\end{enumerate}

\textbf{Bevis.} Lad $M=(Q, \Sigma, \delta, q_0, F)$ være en DFA med $L(M)=A$ og lad pumpe-længden $p=|Q|$. Vi vælger en vilkårlig streng $s \in A$, hvor $|s|=n\ge p$. Når $M$ læser $s$ vil der altså være $n$ præcis $n$ state-transitions og derfor vil sekvensen af states som $M$ befinder sig i mens $s$ læses have længde $n+1$ (pga. vi er i state $q_0$ før det første symbol i $s$ læses). Da $|Q| = n$ følger det af pigeonhole princippet at $\exists q \in Q$ s.t. $q$ besøges mindst to gange i løbet af læsningen af $s$. Illustration:    
\begin{center}
	\includegraphics[width=0.65\textwidth]{pumpingLemma.JPG}
\end{center}
Som det kan ses på tegningen, kan vi skrive $s=xyz$, hvor $x$ er den del af $s$ som er indtil den state som gentages; $y$ er den del mellem states der gentages; $z$ er den del efter gentagelsen. \\

Opfyldelse af betingelserne: 
\begin{enumerate}
	\item Det er åbenlyst fra illustrationen at `løkken' fra og til den gentagne state kan gennenmløbes et vilkårligt antal gange (inkl. 0 gange). 
	\item $|y|>0$ fordi `løkken' angiver en sekvens, der finder sted mellem to \textit{forskellige} forekomster af den gentagne state. (Bemærk at $|y|>0$ også gælder hvis sekvensen er er $\ldots q_9q_9 \ldots$, altså et self-loop uden mellemliggende states). 
	\item Følger af pigeonhole princippet fordi de første $p+1$ states må indeholde en gentagen state da $p=|Q|$.    
\end{enumerate} 
\begin{flushright}
	\qed
\end{flushright}

\subsection*{Eksempel: Anveldelse af pumping lemmaet}

Vi vil vise at $L=\{a^{k^{2}}| \; k \ge 0\}$ ikke er regulært v.h.a. pumping lemmaet. \\

Antag til modstrid at $L$ er regulært. Lad $p$ være pumpe-længden. Vælg $s=a^{p^{2}} \in L$. Da $|s|=p^2 \ge p$ kan vi skrive $s=xyz$. Men hvis vi betragter $s'=xy^2z$ ser vi at $|s'| \le p^2 + p < p^2+2p+1=(p+1)^2$. Den første ulighed holder fordi $|y| \le p$. Det følger at $s' \notin L$, hvilket er en modstrid, der viser at $L$ ikke er regulært. 
\begin{flushright}
	\qed
\end{flushright}    

